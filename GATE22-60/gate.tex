\let\negmedspace\undefined
\let\negthickspace\undefined
\documentclass[journal,12pt,twocolumn]{IEEEtran}

\usepackage{cite}
\usepackage{amsmath,amssymb,amsfonts,amsthm}
\usepackage{graphicx}
\usepackage{textcomp}
\usepackage{xcolor}
\usepackage{txfonts}
\usepackage{listings}
\usepackage{enumitem}
\usepackage{mathtools}
\usepackage{gensymb}
\usepackage[breaklinks=true]{hyperref}
\usepackage{tkz-euclide} % loads  TikZ and tkz-base
\usepackage{listings}
\usepackage{circuitikz}
\usepackage{graphicx}

%\newcounter{MYtempeqncnt}
\DeclareMathOperator*{\Res}{Res}
%\renewcommand{\baselinestretch}{2}
\renewcommand\thesection{\arabic{section}}
\renewcommand\thesubsection{\thesection.\arabic{subsection}}
\renewcommand\thesubsubsection{\thesubsection.\arabic{subsubsection}}

\renewcommand\thesectiondis{\arabic{section}}
\renewcommand\thesubsectiondis{\thesectiondis.\arabic{subsection}}
\renewcommand\thesubsubsectiondis{\thesubsectiondis.\arabic{subsubsection}}

% correct bad hyphenation here
\hyphenation{op-tical net-works semi-conduc-tor}
\def\inputGnumericTable{}                                 %%

\lstset{
	frame=single,
	breaklines=true,
	columns=fullflexible
}



\newtheorem{theorem}{Theorem}[section]
\newtheorem{problem}{Problem}
\newtheorem{proposition}{Proposition}[section]
\newtheorem{lemma}{Lemma}[section]
\newtheorem{corollary}[theorem]{Corollary}
\newtheorem{example}{Example}[section]
\newtheorem{definition}[problem]{Definition}
\newcommand{\BEQA}{\begin{eqnarray}}
	\newcommand{\EEQA}{\end{eqnarray}}
\newcommand{\define}{\stackrel{\triangle}{=}}
\newcommand\figref{Fig.~\ref}
\newcommand\tabref{Table~\ref}
\bibliographystyle{IEEEtran}
%\bibliographystyle{ieeetr}


\providecommand{\mbf}{\mathbf}
\providecommand{\pr}[1]{\ensuremath{\Pr\left(#1\right)}}
\providecommand{\qfunc}[1]{\ensuremath{Q\left(#1\right)}}
\providecommand{\sbrak}[1]{\ensuremath{{}\left[#1\right]}}
\providecommand{\lsbrak}[1]{\ensuremath{{}\left[#1\right.}}
\providecommand{\rsbrak}[1]{\ensuremath{{}\left.#1\right]}}
\providecommand{\brak}[1]{\ensuremath{\left(#1\right)}}
\providecommand{\lbrak}[1]{\ensuremath{\left(#1\right.}}
\providecommand{\rbrak}[1]{\ensuremath{\left.#1\right)}}
\providecommand{\cbrak}[1]{\ensuremath{\left\{#1\right\}}}
\providecommand{\lcbrak}[1]{\ensuremath{\left\{#1\right.}}
\providecommand{\rcbrak}[1]{\ensuremath{\left.#1\right\}}}
\theoremstyle{remark}
\newtheorem{rem}{Remark}
\newcommand{\sgn}{\mathop{\mathrm{sgn}}}
\providecommand{\abs}[1]{\left\vert#1\right\vert}
\providecommand{\res}[1]{\Res\displaylimits_{#1}}
\providecommand{\norm}[1]{\left\lVert#1\right\rVert}
%\providecommand{\norm}[1]{\lVert#1\rVert}
\providecommand{\mtx}[1]{\mathbf{#1}}
\providecommand{\mean}[1]{E\left[ #1 \right]}
\providecommand{\fourier}{\overset{\mathcal{F}}{ \rightleftharpoons}}
%\providecommand{\hilbert}{\overset{\mathcal{H}}{ \rightleftharpoons}}
\providecommand{\system}{\overset{\mathcal{H}}{ \longleftrightarrow}}
%\newcommand{\solution}[2]{\textbf{Solution:}{#1}}
\newcommand{\solution}{\noindent \textbf{Solution: }}
\newcommand{\cosec}{\,\text{cosec}\,}
\providecommand{\dec}[2]{\ensuremath{\overset{#1}{\underset{#2}{\gtrless}}}}
\newcommand{\myvec}[1]{\ensuremath{\begin{pmatrix}#1\end{pmatrix}}}
\newcommand{\mydet}[1]{\ensuremath{\begin{vmatrix}#1\end{vmatrix}}}
\renewcommand{\abstractname}{Question}

\let\vec\mathbf

	
	\vspace{3cm}
	
	


\newcommand{\permcomb}[4][0mu]{{{}^{#3}\mkern#1#2_{#4}}}
\newcommand{\comb}[1][-1mu]{\permcomb[#1]{C}}

%\IEEEpeerreviewmaketitle

\newcommand \tab [1][1cm]{\hspace*{#1}}
%\newcommand{\Var}{$\sigma ^2$}
\usepackage{amssymb}
\usepackage{amsmath}
\title{
	
\title{GATE 2022 IN 60}
\author{EE23BTECH11213 - MUTHYALA NIKHITHA SRI
}


}
\begin{document}

\maketitle

\textbf{Question:} 
A $1\text{kHz}$ sine wave generator having an internal resistance of $50 \ohm$  generates an open-circuit voltage of $10 V_p$. When a capacitor is connected across the output terminals, the voltage drops to $8 V_p$. The capacitance of the capacitor (in microfarads) is \hfill(GATE IN 2022)\\ \\ 

\textbf{Solution: }

\begin{table}[h]
 	\centering
 	\resizebox{6 cm}{!}{
 		 \begin{tabular}{|c|c|c|}
        \hline
        \textbf{Parameter} & \textbf{Description} & \textbf{Value} \\
        \hline
        $V_i$ & Input voltage & $10V_p$  \\ \hline
        $V_o$ & Output voltage & $8V_p$ \\ \hline
        $R$ &Internal resistance & $50 \ohm$ \\ \hline
        $f$ &Frequency of sine wave & $1\text{kHz}$ \\ \hline
        $\omega$ &Angular frequency & $2\pi f$ \\ \hline
        $C$ & Capicatance of capacitor & ? \\ \hline
        $X_c$ & Reactance of capicator & $\frac{1}{j\omega C}$ \\ \hline
        
    \end{tabular}


 	}
 	\caption{Input Parameters}
    \label{tab:tabnikh_60}
 \end{table}

\begin{figure}[htb]
	\centering
	\begin{circuitikz}
\tikzstyle{every node}=[font=\LARGE]
\draw [](9.25,13.25) to[short, -o] (7.5,13.25);
\draw (9.25,13.25) to[R,l={ \LARGE $R$}] (10.25,13.25);
\draw [](10.25,13.25) to[short] (12,13.25);
\draw (11.5,13.25) to[short, -*] (11.5,13.25);
\draw [](11.5,10.75) to[short, -o] (7.5,10.75);
\draw (11.5,10.75) to[short, -*] (11.5,10.75);
\draw [<->, >=Stealth] (7.5,13) -- (7.5,11)node[pos=0.5, fill=white]{Vi};
\draw [](11.75,13.25) to[short, -o] (14.5,13.25);
\draw [](11.5,10.75) to[short, -o] (14.5,10.75);
\draw [line width=0.7pt, <->, >=Stealth] (14.5,13) -- (14.5,11)node[pos=0.5, fill=white]{Vo};
\draw (11.5,13.25) to[R,l={ \LARGE $1/sC$}] (11.5,10.75);
\end{circuitikz}

	\label{fig:1}
\end{figure}

\begin{align}
V_o &= \frac{X_c}{\sqrt{R^2 + X_c^2}}\cdot V_o \\
\implies 8V_p &= \frac{X_c}{\sqrt{50^2 + X_c^2}}\cdot 10V_p \\
\implies X_c^2 - 1.5625X_c^2 + 2500 &= 0 \\
\implies X_c &= \frac{200}{3} \\
\implies C &= \frac{1}{2\pi\cdot 10^3\cdot \frac{200}{3}} \\
\implies C &= 2.387 \mu F
\end{align}

\end{document}
